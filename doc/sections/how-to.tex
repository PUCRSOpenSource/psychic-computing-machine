\section{Como utilizar o simulador}

Para utilizar o simulador é necessário ter o \emph{Ruby} instalado.

O formato do arquivo de entrada deve ser como no exemplo fornecido:

\begin{lstlisting}
#NODE
n1,00:00:00:00:00:01,192.168.0.2,5,192.168.0.1
n2,00:00:00:00:00:02,192.168.0.3,5,192.168.0.1
n3,00:00:00:00:00:03,192.168.1.2,5,192.168.1.1
n4,00:00:00:00:00:04,192.168.1.3,5,192.168.1.1
#ROUTER
r1,2,00:00:00:00:00:05,192.168.0.1,5,00:00:00:00:00:06,\\ 192.168.1.1,5
#ROUTERTABLE
r1,192.168.0.0,0.0.0.0,0
r1,192.168.1.0,0.0.0.0,1
\end{lstlisting}

Para executar o programa execute o comando:

\begin{verb}
ruby app.rb <topologia> <nodo1> <nodo2> <mensagem>
\end{verb}


