\section{Implementação}
\label{sec:imp}


O simulador foi desenvolvido na linguagem \emph{Ruby}. Sua estrutura
consistem em um conjunto de interfaces que podem ser nodos ou
pertencerem a roteadores, e essas mesmas pertencem a uma rede que são
ligadas através dos roteadores.

A interface possui todas as funções principais que são especializadas
quando se trata de um nodo, mantendo o mesmo nome para que todas as
interfaces sejam tratadas da mesma maneira, independente de ser um nodo
ou pertencer a um roteador. Existem as funções de \emph{ARP} e de
\emph{ICMP} para executar os protocolos, por exemplo \verb!arp_request!
e \verb!icmp_reply!.

Além disso existe uma função para enviar mensagem, que é a que chama as
outras, tanto para um interface de roteador ou um nodo.

